%% Copyright 2006-2013 Xavier Danaux (xdanaux@gmail.com).
%
% This work may be distributed and/or modified under the
% conditions of the LaTeX Project Public License version 1.3c,
% available at http://www.latex-project.org/lppl/.

\documentclass[11pt,a4paper,sans]{moderncv}        % possible options include font size ('10pt', '11pt' and '12pt'), paper size ('a4paper', 'letterpaper', 'a5paper', 'legalpaper', 'executivepaper' and 'landscape') and font family ('sans' and 'roman')

% moderncv themes
\moderncvstyle{banking}                            % style options are 'casual' (default), 'classic', 'oldstyle' and 'banking'
\moderncvcolor{purple}                               % color options 'blue' (default), 'orange', 'green', 'red', 'purple', 'grey' and 'black'

% character encoding
% character encoding
\usepackage[utf8]{inputenc}                       % if you are not using xelatex ou lualatex, replace by the encoding you are using

% adjust the page margins
\usepackage[scale=0.75]{geometry}

% personal data
\name{Lívia}{Griebler Taschetto}
%\title{CV}                                        % optional, remove / comment the line if not wanted
\address{Rua Brasil, 1375}{92310-150 Canoas}{RS} % optional, remove / comment the line if not wanted; the "postcode city" and and "country" arguments can be omitted or provided empty
\phone[mobile]{+55~(51)~99584~6564}                 % optional, remove / comment the line if not wanted
%\phone[fixed]{+55~(51)~3059~7423}                  % optional, remove / comment the line if not wanted
\email{liviagriebler@hotmail.com}                       % optional, remove / comment the line if not wanted
%\homepage{taschetto.wordpress.com}                 % optional, remove / comment the line if not wanted
%\extrainfo{31 anos}                                % optional, remove / comment the line if not wanted
%\social[linkedin]{taschetto}
%\social[github]{taschetto}

%----------------------------------------------------------------------------------
%            content
%----------------------------------------------------------------------------------
\begin{document}

\makecvtitle

\section{Objetivo Profissional}
\cvitem{}{Professora de Educação Infantil e Anos Iniciais do Ensino Fundamental.}

\section{Princípios Orientadores de Minhas Práticas Pedagógicas}
\cvitem{}{
\begin{itemize}
\item Reconhecer, valorizar e promover a singularidade, a identidade e as bagagens socioculturais de cada aluno;
\item Promover a expressão dos alunos das mais variadas formas;
\item Realizar planejamentos didático-pedagógicos abertos, flexíveis e sujeitos a alterações conforme as necessidades/dilemas/complexidades do ambiente escolar e seus atores;
\item Promover aulas dinâmicas, interessantes, partindo das inúmeras possibilidades, tais como as metodologias ativas;
\item Reconhecer a importância do currículo, mas assumir que as práticas pedagógicas não devem restringir-se aos conhecimentos naquele restritos;
\item Refletir com criticidade e embasamento teórico acerca das práticas docentes, em busca de aprimoramento contínuo.
\end{itemize}}

\section{Formação Acadêmica}
\cventry{2016}{Especialização em Psicopedagogia}{Pontifícia Universidade Católica do Rio Grande do Sul}{Porto Alegre}{Pós-Graduação}{}
\cventry{2010--2015}{Licenciatura em Pedagogia}{Universidade Federal do Rio Grande do Sul}{Porto Alegre}{}{}
\cventry{2011--2012}{Programa de Intercâmbio UFRGS-UQÀM}{Université du Québec à Montréal}{Montréal - QC - Canadá}{}{}

\section{Experiências Profissionais}
\cventry{2017--2018}{Professora do 2º Ano do Ensino Fundamental}{Colégio La Salle}{Esteio}{}{}
\cventry{2016--2017}{Professora do Turno Integral – E.I e Anos Iniciais E.F}{Colégio Marista Champagnat}{Porto Alegre}{}{}
\cventry{2015--2016}{Professora de Educação Infantil}{Espaço Girassol}{Canoas}{}{}
\cventry{2014}{Auxiliar (Volante) nas turmas Berçário, Nível 1, Nível 2 e Nível 4}{Colégio Farroupilha}{Porto Alegre}{}{}
\cventry{2013--2014}{Auxiliar de Professora nas Classes de Alfabetização/1º Ano do E.F}{Escola Pró-Saber}{Porto Alegre}{}{}

\section{Idiomas}
\cvitemwithcomment{Português}{Nativo}{}
\cvitemwithcomment{Inglês}{Intermediário}{Leitura, escrita e fala.}
\cvitemwithcomment{Francês}{Intermediário}{Leitura, escrita e fala.}

\section{Informática}
\cvitem{}{Proficiência em: Windows; Pacote Office (Word, Excel, PowerPoint); internet.}

\end{document}
