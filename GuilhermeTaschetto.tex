%% Copyright 2006-2013 Xavier Danaux (xdanaux@gmail.com).
%
% This work may be distributed and/or modified under the
% conditions of the LaTeX Project Public License version 1.3c,
% available at http://www.latex-project.org/lppl/.

\documentclass[11pt,a4paper,sans]{moderncv}        % opções: tamanho da fonte ('10pt', '11pt', '12pt'), tamanho do papel ('a4paper', 'letterpaper', etc.), família de fontes ('sans', 'roman')

% temas moderncv
\moderncvstyle{banking}                            % estilos: 'casual', 'classic', 'oldstyle', 'banking'
\moderncvcolor{black}                              % cores: 'blue', 'orange', 'green', 'red', 'purple', 'grey', 'black'

% codificação de caracteres
\usepackage[utf8]{inputenc}

% ajustar margens da página
\usepackage[scale=0.95]{geometry}

% pacotes
\usepackage{enumitem} % para customizar listas

% dados pessoais
\name{Guilherme}{Taschetto}
\title{Engenheiro de Software}
\address{Canoas, Brasil}{}{}
\phone[mobile]{+55~(51)~99288~3831}
\email{gtaschetto@gmail.com}
\social[linkedin]{taschetto}
\social[github]{taschetto}

%----------------------------------------------------------------------------------
%            conteúdo
%----------------------------------------------------------------------------------
\begin{document}

\makecvtitle

\cvitem{}{Minha missão é entregar software de alta qualidade de forma competitiva, escalável e sustentável. Acredito que a melhoria contínua, a colaboração estreita com gerentes de produto e designers, e o compromisso com maturidade e trabalho em equipe são essenciais para alcançar os objetivos e entregar alto valor aos clientes, mantendo-se alinhado aos objetivos de negócio.}

\section{Habilidades}
\cvitem{Linguagens de Programação}{Ruby, JavaScript, TypeScript, C, C++, C\#, Python, Java, Progress/OpenEdge}
\cvitem{Desenvolvimento Web}{Ruby on Rails, Node.js, React, Material UI, Vite}
\cvitem{Cloud e DevOps}{AWS (Lambda, S3, CDK, SQS, CloudFront, API Gateway, CloudFormation, ACM, Route 53), Heroku, CI/CD, Infraestrutura como Código}
\cvitem{Outros}{Desenvolvimento orientado a testes, Feature flags (LaunchDarkly), Metodologias Ágeis (Scrum, Kanban), Git}

\section{Idiomas}
\cvitemwithcomment{Inglês}{Fluente}{}
\cvitemwithcomment{Português}{Nativo}{}

\section{Experiência Profissional}
\cventry{2020--\the\year}{Engenheiro de Software Sênior}{Tatango}{Remoto}{}{
Na Tatango, utilizei um stack que incluía Ruby on Rails, TypeScript, React e Node.js para projetar e implementar funcionalidades tanto no front-end quanto no back-end. Como Tech Lead, fui responsável pela arquitetura da aplicação, utilizando serviços da AWS como CloudFront, API Gateway, Lambda, SQS e DynamoDB para entregar funcionalidades. Minhas principais realizações incluem:
\begin{itemize}[leftmargin=2em]
    \item Implementação de um pipeline de processamento baseado em Lambda para lidar com milhões de mensagens em um curto período, aumentando significativamente a capacidade do sistema.
    \item Criação de um sistema personalizado de encurtamento de URLs capaz de processar picos súbitos de milhões de redirecionamentos para acomodar o tráfego gerado por campanhas massivas de SMS/MMS.
    \item Otimização do processo de desenvolvimento ao introduzir feature flags, resultando em pull requests menores e ciclos de desenvolvimento mais rápidos.
    \item Liderança na melhoria do pipeline de deploy, passando de deployments quinzenais para várias vezes por dia, acelerando o tempo de entrega.
    \item Desenvolvimento de uma ferramenta interna de linha de comando para automatizar tarefas diárias, aumentando a eficiência da equipe de desenvolvimento.
\end{itemize}
}

\cventry{2018--2020}{Engenheiro de Software Sênior}{eGrid}{Porto Alegre}{}{
Na eGrid, liderei o desenvolvimento de um sistema de gestão de energia elétrica, supervisionando a concepção do produto, design e implementação. Minhas responsabilidades incluíram a criação de protótipos, definição de requisitos de usuário e garantia de alta disponibilidade e desempenho. O stack tecnológico incluía Ruby on Rails, TypeScript, React, PostgreSQL e Redis, com infraestrutura em Heroku e AWS. Principais realizações incluem:
\begin{itemize}[leftmargin=2em]
    \item Desenvolvimento de um engine de cálculo flexível para auditoria de faturas de energia elétrica, capaz de processar faturas de todos os 27 estados brasileiros.
    \item Parceria com gerentes de produto para criar relatórios detalhados e ricos em informações para usuários em diversos níveis hierárquicos, melhorando a apresentação de dados e a tomada de decisões.
    \item Integração de múltiplas fontes de dados como o comitê do mercado livre de energia do Brasil, aprimorando a interoperabilidade e confiabilidade do sistema.
    \item Implementação de algoritmos de OCR usando AWS Lambda para automatizar a extração de dados de arquivos PDF, reduzindo significativamente a entrada manual de dados.
\end{itemize}
}

\cventry{2014--2018}{Engenheiro de Software Sênior}{Fator 7}{Porto Alegre}{}{
Desenvolvimento de sistemas ERP usando Ruby on Rails e Progress/OpenEdge.
}

\cventry{2009--2014}{Engenheiro de Software}{Elipse Software}{Porto Alegre}{}{
Desenvolvimento de drivers de comunicação para protocolos industriais e sistemas de aquisição de dados usando C++ e C\#. Utilização de ferramentas como Microsoft Visual Studio e Team Foundation Server.
}

\cventry{2003--2008}{Engenheiro de Software}{TOTVS (anteriormente Datasul)}{Porto Alegre}{}{
Desenvolvimento de soluções ERP personalizadas usando Progress/OpenEdge, XHTML, CSS, JavaScript e Ajax. Responsável pelo design de banco de dados, documentação de software e desenvolvimento de soluções corporativas para internet/intranet.
}

\section{Formação Acadêmica}
\cventry{2012--2016}{Bacharelado em Ciência da Computação}{PUCRS}{Porto Alegre}{}{}
\cventry{2004--2009}{Bacharelado em Engenharia Elétrica}{PUCRS}{Porto Alegre}{}{}

\end{document}
